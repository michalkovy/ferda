% Diplomová práce Michala Kováče
\documentclass{article}
\usepackage{czech}

\author{Michal Kováč}

\begin{document}
\section{O čem prezentace bude}
\section{Motivace}
přepoužít vzory
jednoduché programy - dělat něco v cyklu, 
dělat si vlastní funkce - vymyslet jednoduchý příklad
\section{Dosavadní příspěvky}
od začátku navrženo obecně jako funkce
v designu se počítalo s inteligentníma krabičkama
nicméně někde se zapomnělo/nemyslelo - dynamické property, výstupem množina krabiček, BoxInfo vs functions
přeimplementování GUHA Tomášem Kuchařem -- spojení sémantiky a syntaxe -- mohlo být ale ještě dále, bohužel výkon jde proti obecnosti 
\section{Síťový archiv}
Martin udělal FrontEnd část
co přináší: přepoužití, sdílení
jak ho později rozšířit:
uživatelé a práva
více síťoých archivů
ukládání do souborů
jiné alternativy: include jako jinde v program jazycich, načítání krabek z jiného projektovéh souboru 
\section{Lambda}
co to je za funkci
co s ní de v lambda kalkulu
o co je silnější u nás (rekurze) - příklad jak moc zjednodušuje to zápis
\section{vlastnosti funkcionálního programování}
lazy vyhodnocování
problém zastavení výpočtu, když je program zacyklen
\section{sety a seznamy}
foreach
zapojení setu podobně jako krabičku group do zásuvky
krabička na převod seznamu do setu a naopak
ruční tvorba seznamu - head/tail, vyčíslení prvků
krabičky:
if-then-else (tedy spíše a?b:c)
aritmetika - +,*,-,:,mocnina,<<,>>,and,or,xor
show message
write to file
read value of property?
\section{Příklady setu}
\section{Příklady na Guze}
úprava base/founded implication dle počtu výsledků.
guha mining nad neznámými krabičkami
\section{Programovatelná krabička}
Uživatel vybere počet a typ zásuvek, typ výsleadku a v pythonu dopíše implementaci.
\end{document}
