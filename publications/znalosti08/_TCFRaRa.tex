\documentclass{llncs}
\usepackage{url}
\usepackage{array}
\usepackage{graphicx}
\usepackage{graphics}
\usepackage{amssymb}

%\newtheorem{definition}{Definition}

\begin{document}

\pagestyle{empty}
\mainmatter

\title{Investigating Measures of Association by Graphs and Tables of Critical Frequencies}
\author{Martin Ralbovsk\'{y}, Jan Rauch}

\institute{University of Economics, Prague \\
 W. Churchill Sq.~4,  130 67 Praha~3, Czech Republic \\
\email{martin.ralbovsky@gmail.com, rauch@vse.cz} }

\maketitle

\begin{abstract}
There is lot of effort to find suitable measures of interestingness of association rules. 
The most known measures are confidence and support but there are tens of additional ones. 
Each association measure can be understood as a function of four independent variables. 
These variables correspond to frequencies from a fourfold contingency table of antecedent and succedent. 
A natural way to investigate functions is to study their graphs. However it is hard to deal 
with graphs of functions of four independent variables; thus graphs are generally not suitable 
to study association measures. We show that tables of critical frequencies can be used 
to overcome this difficulty for some important classes of association measures. We give 
an overview of important classes of association measures and then we show how 
the graphs of tables of critical frequencies describe behavior of corresponding association 
measures in a reasonable way. 
\end{abstract}


\section{Introduction}  \label{Introduction}

%-------------------------------------------------------------------------


There are lot of papers dealing with various aspects of association measures; 
see e.g. \cite{Han:07,HeCr,TKS:02}. The goal is to find optimal criterion 
of truthfulness of the association rule $\varphi \approx \psi$ 
expressing a relation of Boolean attributes $\varphi $ and $ \psi$ in given data. 
We suppose to have data matrix $\cal M$ with two Boolean columns corresponding 
to $\varphi $ and $ \psi$. Both $\varphi $ and $ \psi$ can be derived from the other, usually 
non Boolean,  columns.  
The whole situation is fully described by the four-fold 
contingency table 
$4ft(\varphi, \psi, {\cal M})$ (the {\em 4ft table} for short) of $\varphi $ and $\psi$ 
in the data matrix $\cal M$. It is the quadruple $\langle a,b,c,d \rangle $ 
of natural numbers such that $a $ is the number of rows of ${\cal M}$ satisfying both $ \varphi $
and $ \psi $, $b $ is the number of rows of ${\cal M}$ satisfying $ \varphi $
and not satisfying $ \psi $ etc.; see Table \ref{4ft_Table_fi_psi}.
%
\begin{table}[htbp]
\begin{center}
\caption{4ft table $4ft(\varphi, \psi, {\cal M})$ of $\varphi $ and $\psi$ in $\cal M$ } \label{4ft_Table_fi_psi}
\begin{tabular}{r|c|c}
   $ {\cal M} $       & $ \psi $ &  $ \neg \psi $ \\
\hline
     $  \varphi $  & \ \ \  $ a \ \ \  $    & $ \ \ \ b \ \ \ $      \\
\hline
   $ \neg \varphi $  &  $ c $    & $ d $    \\
%\hline
%          &  $ k $    & $ l $  & $n$   \\
\end {tabular}
\end{center}
\end{table}


There are 
 various requirements concerning relation of $\varphi $ and $ \psi$. We can try 
to express the "classical" association rule with confidence and support, some 
relation described by  a simple condition concerning frequencies  
$ a, b, c, d $   from the 4ft table  $4ft(\varphi, \psi, {\cal M})$ or 
even a relation corresponding to a statistical hypothesis test. 
We call the symbol $ \approx $ as {\em 4ft-quantifier} \cite{Ra:05A}.  
The truthfulness  of the rule  $\varphi \approx \psi$ in data matrix 
$\cal M$ is often defined such that $\varphi \approx \psi$ is true if and only if 
$ F_{\approx}(a,b,c,d) \geq p $ or $F_{\approx}(a,b,c,d) \leq p $ where 
$\langle a,b,c,d \rangle = 4ft(\varphi, \psi, {\cal M})$. The function ${\cal F}_{\approx}$ 
and the parameter $p$ are given by the 4ft-quantifier $ \approx $. We call the function
$F_{\approx}$  the {\em evaluation function of 4ft-quantifier} $\approx$. 
Examples of 4ft-quantifiers and their evaluation functions are in section 
\ref{Evaluation_Functions}. 

Properties of the 4ft-quantifier $\approx$ are fully described by behavior of its 
evaluation function $F_{\approx}$. 
A natural way to investigate functions is to study their graphs. 
Some results in this direction are presented in section \ref{Evaluation_Functions}. 
The evaluation function is the function of four independent variables;
thus graphs are not suitable to study association measures. 
This problem can be partly solved by graphs of tables of critical frequencies 
that describe behavior of important 4ft-quantifiers in a comprehensive way;
see section \ref{Classes_Tables}.  Graphs of tables of critical frequencies are used in section 
\ref{Graphs_Tables}  to describe behavior of several important 4ft-quantifiers. Section \ref{RelatedWork}
compares our approach to related work. 
Conclusions and further research are presented in section \ref{Conclusions}. 
 

\input 2EvalFun.tex 
%\section{Evaluation Functions of 4ft-quantifiers} \label{Evaluation_Functions}
% JR po dohode s MR ktere quantifikatory 

\input 3Graphs.tex
%\section{Graphs of Evaluation Functions} \label{Graphs}
%MR 

\input 4TableCF.tex
%\section{Classes of 4ft-quantifiers and Tables of Critical Frequencies} \label{Classes_Tables}


\input 5GraTCF.tex
%\section{Graphs of Tables of Critical Frequencies} \label{Graphs_Tables}

\input 7RelWork.tex

\input 6Conclus.tex
%\section{Conclusions} \label{Conclusions}

%\bibliographystyle{IEEEbib}

\begin{thebibliography}{1} %

\bibitem{Faw}
Fawcett T.: ROC Graphs: Notes and Practical Considerations for Data Mining Researchers.
Technical report HPL-2003-4 HP Labs, 2003

\bibitem{Ha:83}
H\'{a}jek P, Havr\'{a}nek T, Chytil M (1983) GUHA Method. Academia, Prague (in Czech)

\bibitem{Ha:78}
H\'{a}jek P, Havr\'{a}nek T (1978) Mechanising Hypothesis Formation
- Mathematical  Foundations  for  a   General  Theory. Springer, Berlin Heidelberg New York

\bibitem{Kupka}
Kupka D.: \emph{User support 4ft-miner procedure for data mining}. Master Thesis,
Faculty of Mathematics and Physics, Charles University, Prague 2007 (in Czech)

\bibitem {Han:07}
Wu T., Chen Y., Han J.: Association Mining in Large Databases: A Reexamination of Its Measures.
In: Kok J. (Edt.): Knowledge Discovery in Databases: PKDD 2007, conference proceedings. 
Springer Verlag, 2007, ISBN: 3-540-74975-6

\bibitem {HeCr}
{C{\'e}line H. Cr{\'e}milleux} B.: A Unified View of Objective Interestingness Measures. 
In.: Perner P.(Edt.): Machine Learning and Data Mining in Pattern Recognition. 
               MLDM 2007, conference proceedings, Springer, 2007

\bibitem {TKS:02}
Tan P., Kumar V. Srivastava J.: Selecting the right interestingness measure for association patterns. 
In Proceedings of the Eight A CM SIGKDD   International Conference on Knowledge Discovery and Data Mining, July 2002.
    
\bibitem{Ra:86} Rauch J (1986) Logical Foundations of Hypothesis Formation from Databases. 
Mathematical Institute of the Czechoslovak Academy of Sciences, Prague, Dissertation  (in Czech)

\bibitem {Ra:98A}
Rauch J (1998) Classes of Four-Fold Table Quantifiers. 
In: Zytkow J, Quafafou M (Eds.): Principles of Data Mining and Knowledge Discovery. 
Springer, Berlin Heidelberg New York

\bibitem{Ra:04}
Rauch J (2004) Definability of Association Rules and Tables of
Critical Frequencies. In: Lin T Y et al.  (Eds.): Foundations of 
Data Mining. Brighton,  IEEE Computer Society

\bibitem{Ra:05A}
Rauch J (2005) Logic of Association Rules. Applied Intelligence, 
 22, pp. 9-28 

\bibitem{RS:05A} 
Rauch J., \v{S}im\accent23unek M.: (2005) An
Alternative Approach to Mining Association Rules. In:  Lin T Y et al. (Eds.) 
Data Mining: Foundations, Methods, and Applications. 
Springer, Berlin Heidelberg New York

\bibitem{Ra:07}
Rauch, J.: Observational Calculi, Classes of Association Rules and F-property. 
San Francisco 02.11.2007 � 04.11.2007. In: Granular Computing
2007. Los Alamitos: IEEE Computer Society Press, 2007, s. 287�293. ISBN
0-7695-3032-X
	
\bibitem{RK:07}
Ralbovsk\'{y} M, Kucha\v{r} T (2007) Using Disjunctions in 
Association Mining. In: Perner P (Ed.): 
Advances in Data Mining - Theoretical Aspects and Applications.  
Springer, Berlin Heidelberg New York

\end{thebibliography}

\end{document}




\bibitem {Ag:96}
Aggraval R et al. (1996) Fast Discovery of Association Rules.
In: Fayyad, U. M. et al.(Eds.) Advances in Knowledge Discovery and Data
Mining. AAAI Press %/ The MIT Press  

\bibitem{Ha:78A} H\'{a}jek P (guest ed.) (1978) International Journal of
Man-Machine Studies, special issue on GUHA, 10

\bibitem{Ha:81}
H\'{a}jek P (guest ed.) (1981) International Journal of
Man-Machine Studies, second special issue on GUHA, 15

\bibitem{Ha:83}
H\'{a}jek P, Havr\'{a}nek T, Chytil M (1983) GUHA Method. Academia, Prague (in Czech)

\bibitem{Ha:95}
H\'{a}jek P, Sochorov\'{a} A, Zv\'{a}rov\'{a} J (1995) GUHA for personal computers.
Computational Statistics \& Data Analysis,  19

%\bibitem{LL:00}
%Louie E,  Lin T Y (2000) Finding Association Rules using Fast Bit
%Computation: Machine-Oriented Modeling. In: Ras Z, Ohsuga S (eds)
%Foundations of Intelligent Systems. Springer, Berlin Heidelberg New York

\bibitem{PKS:04}
Pang-Ning T, Kumar V, Srivastava J (2004)
Selecting the Right Objective Measure for Association Analysis.  Information Systems, 29

\bibitem{RK:07}
Ralbovsk\'{y} M, Kucha\v{r} T (2007) Using Disjunctions in 
Association Mining. In: Perner P (Ed.): 
Advances in Data Mining - Theoretical Aspects and Applications.  
Springer, Berlin Heidelberg New York


\bibitem{Ra:75}
Rauch J (1975) Ein Beitrag zu der GUHA Methode in der dreiwertigen Logik. Kybernetika, 11

%\bibitem {Ra:78}
%Rauch J (1978) Some Remarks on Computer  Realisations of GUHA
%Procedures. International Journal of Man-Machine Studies 10: 23--28

\bibitem{Ra:86} Rauch J (1986) Logical Foundations of Hypothesis Formation from Databases. 
Mathematical Institute of the Czechoslovak Academy of Sciences, Prague, Dissertation  (in Czech)


%\bibitem{Ra:97}
%Rauch J (1997) Logical Calculi for Knowledge Discovery in Databases.
%In: Zytkow J, Komorowski J (Eds.): Principles of Data Mining and Knowledge Discovery. 
%Springer, Berlin Heidelberg New York


%\bibitem {Ra:98B}
%Rauch J (1998) Four-Fold Table Calculi and Missing Information.
%in Proc. Joint Conference on Information Sciences, Durham, North Carolina, 1998, pp. 375-378


\bibitem {Ra:98C}
Rauch J (1998) Contribution to Logical Foundations of KDD. 
University of Economics Prague, Assoc. Prof. Thesis (in  Czech)

%\bibitem{Ra:02A}
%Rauch J (2002) Interesting Association Rules and Multi-relational Association Rules.
%Communications of Institute of Information and Computing Machinery, Taiwan,
%Vol. 5, pp. 77-82, May 2002


%\bibitem{Ra:05B}
%Rauch J (2005) Definability of Association Rules in Predicate
%Calculus.  In:  Lin T Y, Ohsuga S,  Liau C J,  Hu X (eds)
%Foundations and Novel Approaches in Data Mining. Springer, Berlin
%Heidelberg New York pp. 23 - 40

\bibitem{Ra:05C}
Rauch J (2005) Classes of Association Rules, an Overview.
In: Lin T Y, Xie Y (Eds.). Foundation of semantic Oriented Data
and Web Mining. %[online] 
Houston, IEEE Computer Society
%, 2005, pp. 68�-74. ISBN 0-9738918-7-4.
%\url{http://www.cs.sjsu.edu/faculty/tylin/ICDM05/}.


%\bibitem{RSL:05}
%Rauch Jan, \v{S}im\accent23unek Milan, L\'{i}n V\'{a}clav (2005)
%Mining for Patterns Based on Contingency Tables by KL-Miner � First Experience.%
%In: LIN, Tsau Young, OHSUGA, Setsuo, LIAU, C. J., HU, Xiaohua (Eds.)
% Foundations and Novel Approaches in Data Mining. Berlin : Springer-Verlag, pp. 155 -- 167. ISBN 3-540-28315-3


\bibitem{RS:05B}
Rauch J, \v{S}im\accent23unek M (2005)
GUHA Method and Granular Computing. In: Hu, X et al. (Eds.) 
%et iaohua, LIU, Qing, SKOWRON, Andrzej, LIN, Tsau Young, YAGER, Ronald R., ZANG, Bo (ed.). 
Proceedings of IEEE conference Granular Computing. Beijing, IEEE Computer Society

\bibitem{RS:07B}
Rauch J, \v{S}im\accent23unek M: Semantic Web Presentation of Analytical 
Reports from Data Mining  -- Preliminary Considerations. In Proceedings of Web 
Intelligence 07

%\bibitem {Ze:96}
%R. Zembowicz and J. Zytkow (1996) From Contingency Tables to Various Forms of
%Knowledge in Databases. in Advances in Knowledge Discovery and Data
%Mining, edited by U. M. Fayyad and all, AAAI Press: Menlo Park, California, pp. 329 - 349, 1996

