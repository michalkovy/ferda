\section{Conclusions} \label{Conclusions}
Understanding of association measures is one of the key factors in association mining\cite{Han:07,HeCr,TKS:02}. The subject has been gaining lasting scientific attention. In the paper, we studied graphs of evaluation functions of association measures (quantifiers). This task was difficult mainly because the evaluation functions are in general functions of four independent variables. However, for some quantifiers this was possible. We constructed graphs of \emph{founded equivalence} and \emph{pairing} quantifiers, the graphs enabled comparison of the two quantifiers and showed some properties, which were not apparent without the graphical representation. 

For other quantifiers, we used existing theory concerning classes of quantifiers and \emph{tables of critical frequencies}. Tables of critical frequencies help to reduce the dimensionality of the quantifier evaluation function while preserving properties of the quantifier. We compared \emph{founded implication} with \emph{lower} and \emph{upper} critical implication from the implicational class by constructing \emph{tables of maximal b's}. For symmetrical quantifiers with F-property one can construct \emph{tables of minimal $|b-c|$}. We constructed the tables for \emph{Fisher's} quantifier, \emph{above average dependence} and \emph{simple deviation} and observed interesting properties, the most significant being the \emph{F-strength} of all observed quantifiers along the inverse $ad$ diagonal.

To summarize, graphing functions of quantifiers combined with theoretical result is a strong and promising tool that broadens our knowledge about quantifiers. The paper is an initiatory work in the field, majority of quantifiers and possibilities to graph them were unemployed. The work also showed further direction of research in the area. Issues such as construction of graphs of critical frequencies for other classes of quantifiers, effects of changes of the parameters on the graphs or various combinations of several graphs will be subjected to further research. 

\subsection*{Acknowledgements}
This work was supported by the project MSM6138439910 of the Ministry of Education of the Czech Republic and by the project 201/05/0325 of the Czech Science Foundation.