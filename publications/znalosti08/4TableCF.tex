\section{Tables of Critical Frequencies} \label{Classes_Tables}

Tables of critical frequencies are closely related to classes of 4ft-quantifiers.  
Examples of classes of 4ft-quantifiers are: {\em implicational quantifiers} \cite{Ha:78}, 
{\em double implicational quantifiers} \cite{Ra:05A}
or {\em $\Sigma$- equivalency quantifiers} \cite{Ra:05A}. 
There are both theoretically interesting and practically important results 
that are related to classes of 4ft-quantifiers. These results concern
namely deduction rules  of the form  $ \frac{\varphi \approx \psi}{\varphi' \approx \psi'} $
where both $ \varphi \approx \psi $ and 
$ \varphi' \approx \psi' $ are association rules; see \cite{Ra:05A}, dealing with missing information; 
see \cite{Ha:78,Ra:98A} and definability of association rules in classical predicate calculi 
with equality \cite{Ra:04}. There are also results concerning tables of critical frequencies that 
can be used to optimize verification of particular association rules. 

We use tables of critical frequencies to simplify graphical description 
of behavior of some of quantifiers introduced in section 
\ref{Evaluation_Functions}. We deal with the class of {\em implicational quantifiers}
and with the class of {\em quantifiers with F-property}.  

The class of {\em implicational quantifiers} is defined in \cite{Ha:78} such that 
the 4ft quantifier $ \Rightarrow^*  $ is implicational if it satisfies the condition 
%
$$ \Rightarrow^*(a,b,c,d) = 1   \ \ \land \ \ a' \geq a \  \land b' \leq b \ \mbox{ implies} \Rightarrow^*(a,b,c,d)  = 1 $$
%
for all the 4ft tables $ \langle a,b,c,d \rangle $   and $ \langle a',b',c',d'\rangle $.
It is proved in \cite{Ha:78}  that the quantifiers 
 $ \Rightarrow_{p} $ of {\it founded implication \/},  $ \Rightarrow^{!}_{p, \alpha} $ of
 {\it lower critical implication} and 
 $ \Rightarrow^{?}_{p, \alpha} $ of
 {\it upper critical implication} (see section 
 \ref{Evaluation_Functions}) are implicational. 
It is easy to prove that for each implicational quantifier there is a 
non-negative and non-decreasing function
$ Tb_{\Rightarrow^{*}}$ with value
\mbox{ $ Tb_{\Rightarrow^{*}}(a) \in  \{ 0,1,2, \dots \} \cup \{ \infty \}$}
such that it is 
$$ \Rightarrow^{*}(a,b) = 1  \ \mbox{ if and only if}  \ b <  Tb_{\Rightarrow^{*}}(a) \  $$
for all integers $ a \geq 0$ and $ b \geq 0 $.
We call the function $Tb_{\Rightarrow^{*}}$
a {\it table of maximal b for implicational quantifier \/} $\Rightarrow^{*}$ \cite{Ha:78,Ra:98A}.
It is important that precomputed function $Tb_{\Rightarrow^{*}}$
makes it possible to use a simple test of inequality
instead of a rather complex computation. E.g.,  we can use inequality
\mbox{ $ b < Tb_{\Rightarrow^{!}_{p, \alpha}}(a) $}
instead of condition \mbox{ $ \sum_{i = a}^{a + b} \frac{(a+b)!}{i!(a+b-i)!}
          p^{i} (1 - p)^{a+b-i} \leq \alpha $}
for quantifier
$ \Rightarrow^{!}_{p, \alpha} $ of lower critical implication. 

The class of {\em 4ft-quantifiers with F-property} is defined in \cite{Ra:86} (see  also 
\cite{Ra:07}) such that 
the 4ft quantifier $\approx$ has the {\em F-property} if it satisfies:
\begin{enumerate}
\item If $ \  \approx(a,b,c,d) = 1 $ and $ b \geq c-1 \geq 0 $
then $ \approx(a,b+1,c-1,d) = 1 $.

\item If $ \ \approx(a,b,c,d) = 1 $ and $ c \geq b -1 \geq 0 $
then $ \approx(a,b-1,c+1,d) = 1 $.
\end{enumerate}

We \ say \ that \ the \ quantifier \ $\approx$ \ is \ {\em symmetrical} \ \cite{Ha:78}
if \ it \ satisfies \mbox{$\approx(a,b,c,d) = \approx(a,c,b,d)$}. 
It is proved in \cite{Ra:86} that for the symmetrical 4ft-quantifier $\approx$ with the 
F-property there is a function $T_{\approx}$ that assigns to each triple $\langle a,d,n \rangle$ of natural numbers satisfying 
$a+d \leq n $ the number $T_{\approx}(a,d,n)$ such that for each $b \geq 0$ and $c \geq 0$ where 
$a+b+c+d = n $ it is 
%
$$ \approx(a,b,c,d) = 1 \mbox{ iff } |b-c| \geq T_{\approx}(a,d,n) \mbox{ .}$$

The function $T_{\approx}(a,d,n)$ can be used in the same way as the function 
$ Tb_{\Rightarrow^{*}}(a) $ for the implicational quantifier 
$ \Rightarrow^{*}$,  see above. 
The function $T_{\approx}(a,d,n)$ is called {\em table of minimal} 
$|b-c|$. 

The  functions $Tb_{\Rightarrow^{*}}$ and $T_{\approx}(a,d,n)$ are called 
{\em tables of critical frequencies}. Note that there are also tables of critical frequencies 
for additional classes  of 4ft-quantifiers; see e.g. \cite{Ra:07}. 

The above stated theory allows us to compare quantifiers with reduced dimensionality. For given implicational quantifiers $\Rightarrow_{1}$ and $\Rightarrow_{2}$ and for given $a$, we say that $\Rightarrow_{1}$ \emph{is I-stronger than} $\Rightarrow_{2}$, if $Tb_{\Rightarrow_{1}}(a) > Tb_{\Rightarrow_{2}}(a)$. Similarly, for given quantifiers with the F-property $\approx_{1}$, $\approx_{2}$ and for given $a$, $d$ and $n$, we say that $\approx_{1}$ \emph{is F-stronger than} $\approx_{2}$, if $T_{\approx_{1}}(a,d,n) < T_{\approx_{2}}(a,d,n)$.