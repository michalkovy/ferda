\section{Conclusions and Future Work} \label{Conclusions}

The approach to ontology support to association mining implemented in the Ferda tool focuses on providing support for the data preparation phase.
Information about important data values and data column groupings, once injected into a domain ontology, can be repeatedly used for creating meaninfgul categories for attributes and for defining mining tasks producing association hypotheses well-interpretable in the domain context.
Tests on real data have been carried out in the domain of cardiology.

In the future, we plan to extend the support to further \emph{phases} of the association mining process, as already envisaged in \cite{Ralbovsky} and \cite{Ontology}.
We will also enhance the method of \emph{introducing} data-related knowledge to ontologies, among other reflecting the evolution of the OWL language.
It will be necessary to balance the accuracy of data-to-ontology mapping (where mapping to datatype properties seems to be most relevant) with the ergonomy for the end user (who might prefer a concept-centric view of ontology).
Finally, it is likely that not only the most classical approach to GUHA-based mining (relying on four-fold table quantifiers) but also other \emph{mining methods} implemented in Ferda could benefit from exploiting ontological knowledge.