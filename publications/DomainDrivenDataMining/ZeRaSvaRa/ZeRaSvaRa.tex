\documentclass{llncs}
\usepackage{url}
\usepackage{array}
\usepackage{graphicx}

\begin{document}

\pagestyle{empty}
\mainmatter

\title{Ontology-Driven Data Preparation for Association Mining}

\author{Martin Zeman\inst{1}, Martin Ralbovsk\'{y}\inst{2}, Vojt\v{e}ch Sv\'{a}tek\inst{2}, Jan Rauch\inst{2}}
\institute{Department of Software Engineering, Faculty of Mathematics and Physics\\
Charles University, Malostransk\'{e} n\'{a}m. 25, 118 01 Prague, Czech Republic
\email{martinzeman@email.cz}
\and
Department of Information and Knowledge Engineering,\\
University of Economics, Prague, W. Churchill Sq.~4, 130 67 Praha~3, Czech Republic\\
\email{ralbovsm@vse.cz, svatek@vse.cz, rauch@vse.cz}
}

\maketitle

\begin{abstract}
Ontologies can convey domain semantics to various phases of a KDD application through a mapping established between ontology entities and columns of the data matrix.
The approach implemented in the Ferda tool focuses on providing support for the data preparation phase.
Information about important data values and data column groupings, once injected into a domain ontology, can be repeatedly used for creating meaninfgul categories for attributes and for defining mining tasks producing association hypotheses well-interpretable in the domain context.
Tests on real data have been carried out in the domain of cardiology.
\end{abstract}

\input 1Introduction.tex

\input 2OntologiesAssociation.tex

\input 3Ferda.tex

\input 4OntologiesFerda.tex

\input 5Experiments.tex

\input 6Related.tex

\input 7Conclusions.tex

\section*{Acknowledgment}

This work was supported by the CSF project no.201/08/0802, ``Application of Knowledge Engineering Methods in Knowledge Discovery from Databases''.

\begin{thebibliography}{1}

\bibitem{Hypertension}
1999 World Health Organization � International Society of Hypertension Guidelines for the Management of Hypertension. Guidelines Subcommittee. \emph{J~Hypertension}, 1999, 17: 151�-83.

\bibitem{OWL2}
\emph{OWL 2 Web Ontology Language: Profiles}. W3C Working Draft 11 April 2008, online \url{http://www.w3.org/TR/2008/WD-owl2-profiles-20080411/}.

\bibitem{Agrawal}
Agrawal R., Imielinski T., Swami A.:
Mining association rules between sets of items in large databases.
In: Proc. of the ACM SIGMOD Conference on Management of Data, p.207--216.

\bibitem{Hussein} 
Almuallim, H., Akiba, Y. A., Kaneda, S.: On Handling Tree-Structured Attributes in Decision Tree Learning. 
In: Proceedings of the Twelfth International Conference on Machine Learning (ML-95). Morgan Kaufmann, 12--20.

\bibitem{Aronis} 
Aronis, J.M., Provost, F.J., Buchanan, B.G.: Exploiting Background Knowledge in Automated Discovery. 
In: Proceedings of the 2nd International Conference on Knowledge Discovery and Data Mining, 1996 (KDD-96).
%Proc. SIGKDD-96.

\bibitem{Bogorny} 
Bogorny, V., Engel, P., Alvares, L.O.: 
Enhancing the Process of Knowledge Discovery in Geographic Databases using Geo-Ontologies. 
In: Nigro, H. O., Cisaro, S.G., Xodo, D. (Ed.). Data Mining with Ontologies: Implementations, Findings, and Frameworks. Idea Group Inc. (2007). pp.160-181.

\bibitem{Cannataro}
Cannataro, M., Guzzi, P. H., Mazza, T., Tradigo, G., Veltri, P.: Using Ontologies in PROTEUS for Modeling Proteomics Data Mining Applications.
In: From Grid to Healthgrid: Proceedings of Healthgrid 2005, IOS Press, 17-26. 

\bibitem{Clark}
Clark, P. Matwin, S.: Using Qualitative Models to Guide Inductive Learning. 
In: Proceedings of the 1993 International Conference on Machine Learning, 49-56.

\bibitem{Collard}
Brisson, L., Collard, M., Le Brigant, K., Barbry, P.: KTA: A Framework for Integrating Expert Knowledge and Experiment Memory in Transcriptome Analysis. In: International Workshop on Knowledge Discovery and Ontologies, held with ECML/PKDD 2004, Pisa, p.85-90.

%\bibitem{Domingues}
%Domingues, M. A., Rezende S. A.:
%Using Taxonomies to Facilitate the Analysis of the Association Rules.
%In: The 2$^{nd}$ International Workshop on Knowledge Discovery and Ontologies, held with ECML/PKDD 2005, Porto, p.59-66.

\bibitem{Cespivova}
\v{C}e\v{s}pivov\'{a} H. Rauch J., Sv\'{a}tek V., Kejkula M., Tome\v{c}kov\'{a} M.: Roles of Medical Ontologies in Association Mining CRISP-DM Cycle, ECML/PKDD Workshop on Knowledge Discovery and Ontologies (KDO'04), Pisa 2004.

\bibitem{Embley}
Embley, D.~W., Tao, C., Liddle, D.~W.:
\newblock Automatically extracting ontologically specified data from {HTML} tables of unknown structure.
\newblock In {\em Proc. ER '02}, pp. 322--337, London, UK, 2002, Springer-Verlag. 

\bibitem{GUHA}
H\'{a}jek P., Havr\'{a}nek, T.: 
\emph{Mechanising Hypothesis Formation -- Mathematical  Foundations  for  a   General  Theory.}
Springer-Verlag, 1978.

\bibitem{HajekHolena}
H\'{a}jek P., Hole\v{n}a M.: Formal logics of discovery and hypothesis 
formation by machine. \emph{Theoretical Computer Science}, 292 (2003) p.345--357.

\bibitem{Kovac}
Kov\'{a}\v{c} M.:User oriented language for solving KDD tasks. 
Master Thesis, Faculty of Mathematics and Physics, Charles University, Prague (to appear).

\bibitem{Ferda}
Kov\'{a}\v{c} M., Kucha\v{r} T., Kuzmin A., Ralbovsk\'{y} M.: Ferda, 
New Visual Environment for Data Mining. Znalosti 2006, 
Conference on Data Mining, Hradec Kr\'{a}lov\'{e} 2006, p.~118--129 (in Czech).

\bibitem{Kuzmin}
Kuzmin A.: Relational GUHA procedures. Master Thesis, 
Faculty of Mathematics and Physics, Charles University, Prague 2007 (in Czech)

\bibitem{Prickl}
Labsk\'{y}, M., Nekvasil, M., Sv\'{a}tek, V., Rak, D.:
\newblock The Ex Project: Web Information Extraction using Extraction Ontologies. 
\newblock In: Proc. PriCKL'07, ECML/PKDD Workshop on Prior Conceptual Knowledge in Machine Learning and Knowledge Discovery. Warsaw 2007. 

\bibitem{Nunez} 
N\'{u}\~{n}ez, M.: The Use of Background Knowledge in Decision Tree Induction. 
{\em Machine Learning}, 6, 231--250 (1991).

\bibitem{Philips}
Phillips, J., Buchanan, B.G.: Ontology-guided knowledge discovery in databases. 
In: International Conf. Knowledge Capture (K-CAP, 2001), Victoria, Canada, 2001.

\bibitem{Ralbovsky}
Ralbovsk\'{y} M.: Usage of Domain Knowledge for Applications of GUHA Procedures. Master thesis, Faculty of Mathematics and Physics, Charles University, Prague, 2006.

\bibitem{ETree}
Ralbovsk\'{y} M., Berka P.: Implementation of GUHA Decision Trees. In: MIS 2008, Computer science conference, Josef\accent23uv D\accent23ul (to appear).

\bibitem{Disjunctions}
Ralbovsk\'{y} M., Kucha\v{r} T.: Using Disjunctions in Association Mining. 
In: P. Perner (Ed.), Advances in Data Mining - Theoretical Aspects and Applications, LNAI
4597, Springer Verlag, Heidelberg 2007.

\bibitem{Rauch}
Rauch J.: Logic of Association Rules. \emph{Applied Intelligence}, Vol. 22,
Issue 1, p.~9~--~28.

\bibitem{Adamek}
Rauch J., Tome\v{c}kov\'{a} M.: System Of Analytical Questions And Reports on On Mining In Health Data � A Case Study. MCCSIS 2007 [CD-ROM]. Lisabon : IADIS, 2007. pp.~176--181.

\bibitem{Alternative}
Rauch J., \v{S}im\accent23unek, M.: An Alternative Approach to Mining
Association Rules. Lin T Y, Ohsuga S, Liau C J, and Tsumoto S (eds):
Foundations of Data Mining and Knowledge Discovery, Springer-Verlag, 2005
p.~219~--~239.

\bibitem{SEWEBAR}
Rauch, J., \v{S}im\accent23unek, M.: 
Semantic Web Presentation of Analytical Reports from Data Mining---Preliminary Considerations. 
In: Web Intelligence 2007, Los Alamitos: IEEE Computer Society, 2007, pp. 3 � 7.

\bibitem{Svatek}
Sv\'{a}tek, V.: Exploiting Value Hierarchies in Rule Learning. 
In: 
%van Someren, M. - Widmer, G. (Eds.): 
ECML'97, 9th European Conference on Machine Learning. Poster Papers. Prague 1997, 108--117.

\bibitem{Ontology}
Sv\'{a}tek V., Rauch J., Ralbovsk\'{y} M.: Ontology-Enhanced Association
Mining. In: Ackermann, Berendt (eds.). Semantics, Web and Mining, 
Springer-Verlag, 2006.

\bibitem {Simunek}
\v{S}im\accent23unek M.: Academic KDD Project LISp-Miner.
In: 
%Abraham A. et al (eds): 
Advances in Soft Computing---Intelligent Systems Design and Applications, 
Springer Verlag 2003.

\bibitem{Thomas}
Thomas J., Laublet, P., Ganascia, J. G.: A Machine Learning Tool Designed for a Model-Based Knowledge Acquisition Approach. 
In: EKAW-93, European Knowledge Acquisition Workshop, Lecture Notes in Artificial Intelligence No.723, N.Aussenac et al. (eds.), Springer-Verlag, 1993, 123--138.

\bibitem{Someren}
van Dompseler, H. J. H., van Someren, M. W.: Using Models of Problem Solving as Bias in Automated Knowledge Acquisition. 
In: ECAI'94 - European
Conference on Artificial Intelligence, Amsterdam 1994, 503--507.

\bibitem{Zeman}
Zeman M.: Usage of Ontologies for GUHA Procedures. Master thesis, Faculty of Mathematics and Physics, Charles University, Prague 2008.

\bibitem{Zakova}
\v{Z}\'{a}kov\'{a}, M., \v{Z}elezn\'{y}, F.:  Exploiting Term, Predicate, and Feature Taxonomies in Propositionalization and Propositional Rule Learning. ECML 2007: the 18th European Conference on Machine Learning, Springer 2007.

%\bibitem{SWM} Berendt, B., Hotho, A., Stumme, G.: 2nd Workshop on Semantic Web Mining, held at ECML/PKDD-2002, Helsinki 2002,
%\url{http://km.aifb.uni-karlsruhe.de/semwebmine2002}.

%\bibitem{IOS-book}
%Buitelaar, P., Cimiano, P., Magnini, B. (eds.): Ontology Learning and Population, IOS Press, 2005.

%\bibitem{IOS-book}
%Buitelaar, P., Franke, J., Grobelnik, M., Paass, G., Sv\'{a}tek, V. (eds.): ECML/PKDD 2004 Workshop on Knowledge Discovery and Ontologies (KDO-04), Pisa 2004.

%\bibitem{Cespivova}
%\v{C}e\v{s}pivov\'{a}, H., Rauch, J., Sv\'{a}tek V., Kejkula M., Tome\v{c}kov\'{a} M.: Roles of Medical Ontology in Association Mining CRISP-DM Cycle. In: ECML/PKDD04 Workshop on Knowledge Discovery and Ontologies (KDO'04), Pisa 2004.

%\bibitem{OE}
%G\'{o}mez-Perez, A., Fern\'{a}ndez-Lopez, M., Corcho, O.: Ontological Engineering. Springer 2004.

%\bibitem{GUHA}  %%%%%%%%%%%%% DODAL RAUCH
%H\'{a}jek, P., Havr\'{a}nek, T.: \emph{Mechanising Hypothesis Formation - Mathematical  Foundations  for  a   General  Theory}. Springer-Verlag: Berlin  - Heidelberg - New York, 1978.

%\bibitem{swrl}
%Horrocks, I., Patel-Schneider, P. F., Boley, H., Tabet, S., Grosof, B., Dean, M.:
%SWRL: A Semantic Web Rule Language Combining OWL and RuleML. W3C Submission, 21 May 2004. Online \url{http://www.w3.org/Submission/SWRL}.

%\bibitem{Lin}
%L\'{\i}n, V., Rauch, J., Sv\'{a}tek, V.: Content-based Retrieval of Analytic Reports. In: Schroeder, M., Wagner, G. (eds.). Rule Markup Languages for Business Rules on the Semantic Web, Sardinia 2002, 219--224.

%\bibitem{Maedche}
%Maedche, A.: Ontology Learning for the Semantic Web. Kluwer, 2002.

%\bibitem{Trondheim}
%Rauch, J.: Logical Calculi for Knowledge Discovery in Databases. In: Principles of Data Mining and Knowledge Discovery (PKDD-97), Springer-Verlag, 1997.

%\bibitem{Ra:05A}   %%%%%%%%%%%%% DODAL RAUCH
%Rauch, J.: Logic of Association Rules. \emph{Applied Intelligence}, 22, 9�-28, 2005.

%\bibitem{LM}  %%%%%%%%%%%%% OPRAVIL RAUCH
%Rauch, J., \v{S}im\accent23unek, M.: An Alternative Approach to Mining Association Rules. In: Lin, T. Y., Ohsuga, S.,  Liau, C. J., Tsumoto, S. (eds.), Data Mining: Foundations, Methods, and Applications, Springer-Verlag, 2005, pp. 211--232

%\bibitem{SpecVal}
%Rector, A. (ed.): Representing Specified Values in OWL: "value partitions" and "value sets".
%W3C Working Group Note, 17 May 2005, online at \url{http://www.w3.org/TR/swbp-specified-values/}.

%\bibitem{Strossa}  %%%%%%%%%%%%% DODAL RAUCH
%Strossa, P., \v{C}ern\'{y}, Z., Rauch, J.:. Reporting Data Mining Results in a Natural Language. In: Lin, T. Y.,  Ohsuga, S., Liau, C. J.,  Hu, X.  (ed.):  Foundations of Data Mining and Knowledge Discovery. Berlin : Springer, 2005, pp. 347-362

%\bibitem{KDO05}
%Sv\'{a}tek, V., Rauch, J., Flek, M.: Ontology-Based Explanation of Discovered Associations in the Domain of Social Reality. In: The 2$^{nd}$ ECML/PKDD Workshop on Knowledge Discovery and Ontologies, 2005, Porto, 75-86.

\end{thebibliography}

\end{document}